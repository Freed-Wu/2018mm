\clearpage
\section{基于评价和规划的识别模型}
\subsection{模型的建立}
\subsubsection{评价}
声明结构体船,属性有船名、R、L1、L2和A。根据附件3定义所有船\(S_1..S_N\)。
\\\indent 声明结构体R、L1、L2和A,属性为附件4给出的数据。根据附件4定义所有信号源\(A_1..A_{N_1},\\B_1..B_{N_2},C_1..C_{N_3},D_1..D_{N_4}\)。
\\\indent 根据附件5定义船\(s_1..s_n\)和信号源\(a_1..a_{n_1},b_1..b_{n_2},c_1..c_{n_3},d_1..d_{n_4}\)。
\\\indent 指标分为匹配指标和识别指标。
\begin{description}
	\item[匹配指标]必须要完全一样才能视为同一信号源。匹配成功记该指标得分为1,匹配失败记该指标得分为0。
	\begin{align}
		p(i_r(k),i_R(k))=
		\begin{cases}
			1	&	i_r(k)=i_R(k)	\\
			0	&	i_r(k)\neq i_R(k)
		\end{cases}
	\end{align}
	\par \(p\)是得分函数,\(i_r(k),i_R(k)\)为信号源\(r,R\)的第\(k\)个指标,是匹配指标。
	\item[标量识别指标]得分记作两比较对象指标较小的除以指标较大的商,若两比较对象指标相同得分为1,若相差越大得分越小。
	\begin{align}
		p(i_r(k),i_R(k))=
		\begin{cases}
			\dfrac{i_r(k)}{i_R(k)}	&	i_r(k)\leqslant i_R(k)	\\
			\dfrac{i_R(k)}{i_r(k)}	&	i_r(k)\geqslant i_R(k)
		\end{cases}
	\end{align}
	\par \(i_r(k),i_R(k)\)为信号源\(r,R\)的第\(k\)个指标,是标量识别指标。
	\item[向量识别指标]若两比较对象识别指标维度不同,得分记作0;若相同,将该指标拆解为多个不同的标量后作为标量识别指标计算得分,该指标得分记作所有标量识别指标得分之积。
	\begin{align}
		p(\bm{i}_r(k),\bm{i}_R(k))=
		\begin{cases}
			0										&	\mathrm{length}(\bm{i}_r(k))\neq\mathrm{length}(\bm{i}_R(k))		\\
			\prod\limits_{j=1}^{J}p(\bm{i}_r(k)_j,\bm{i}_R(k)_j)	&	\mathrm{length}(\bm{i}_r(k))=\mathrm{length}(\bm{i}_R(k))=J
		\end{cases}
	\end{align}
	\par \(\bm{i}_r(k),\bm{i}_R(k)\)为信号源\(r,R\)的第\(k\)个指标,是向量识别指标。\(\bm{i}_r(k)_j,\bm{i}_R(k)_j\)是信号源\(r,R\)的第\(k\)个指标的第\(j\)个分量。length(\(\bm{i}_r(k)\)),length\(\bm{i}_r(k)\))是信号源\(r,R\)的第\(k\)个指标的维度。
\end{description}
\par 为了避免对\(\dfrac{0}{0}\)进行定义为1的麻烦,所有的0用计算机能表示的最小正数eps取代。
\\\indent 两信号源得分记作两信号源所有指标得分之积。两船舶的得分记作两船舶所有指标得分之积。
\begin{align}
	p(r,R)	&	=\prod\limits_{k=1}^Kp(i_r(k),i_R(k))	\\
	p(s,S)	&	=\prod\limits_{l=1}^Lp(i_s(l),i_S(l))
\end{align}
\par \(r,R\)是信号源,\(s,S\)是船,\(i_r(k),i_R(k)(i_s(k),i_S(k)\)是\(r,R,s,S\)的第\(k\)个指标。
\\\indent 总得分计算亦可使用累加的方式,实际上累加和累乘通过指数和对数一一对应。但综合以下原因选择累乘。
\begin{enumerate}
	\item 选择累加匹配失败得分必须记作\(-\infty\),计算机无法真正表示\(-\infty\);
	\item 累乘可以设定分数上限,比如1,方便比较。
\end{enumerate}
\subsubsection{规划}
用已知的\(N\)艘船舶识别未知的\(n\)艘船舶一共有\(N^n\)种单射,每种单射得分记作所有该单射包含元素得分之积。
\begin{align}
	&	\max p(f)		\\
	&	\mathrm{s.t.}
	\begin{cases}
		p(f)=\prod\limits_{k=1}^np(s_k,f(s_k))		\\
		\{f(s_k)|k=1..n\}\subseteq\{S_k|k=1..N\}		\\
		p(s_k,f(s_k))>0
	\end{cases}
\end{align}
\par 称\(f(s_k)\)为\(s_k\)的一个识别结果,称\((s_k,f(s_k))\)为1个识别对。
\begin{enumerate}
	\item 第1个约束条件是因为识别结果必须是数据库中已知的船;
	\item 第2个约束条件是因为如果单射中有一个元素得分为0,由于连乘的性质,该单射得分为0,必定不是最优解。所以可以通过限制得分不为0的约束条件减少可行解的个数,更加方便地得到最优识别结果。
\end{enumerate}
\par 因为时间复杂度较小,可以通过暴力穷举搜索最优单射。
\begin{align}
	T(N^n)=\Theta(N^n)
\end{align}
\par 但该规划问题存在最优子结构,根据贪心算法只需暴力穷举搜索每艘船舶的最优识别结果,最优单射必是最优识别对的并。
\begin{align}
	&	\max p(f)		\\
	&	\mathrm{s.t.}
	\begin{cases}
		p(f)=\prod\limits_{k=1}^n\max\limits_{f(s_k)\in\{S_k|k=1..N\}}p(s_k,f(s_k))	\\
		p(s_k,f(s_k))>0												\\
	\end{cases}
\end{align}
\par 时间复杂度会进一步下降。
\begin{align}
	T(Nn)=\Theta(Nn)
\end{align}
\subsection{模型的求解}
求解得到信号源识别结果和船只识别结果如下表所示。
\begin{table}[htbp]
	\centering
	\caption{船舶识别结果}
	\begin{tabular}{ccc}
		\toprule
		目标编号		&	船舶名称			&	得分			\\
		\midrule
		101			&	DBE-AY02  		&	0.930799861	\\
		104			&	DBH-SS02     		&	0.940527104	\\
		107			&	DBC-JZ    		&	0.917167393	\\
		110			&	DBE-KDX1-YZWD	&	0.930959406	\\
		113			&	DBH-SS01		&	0.976011509	\\
		\bottomrule
	\end{tabular}
\end{table}
\begin{table}[htbp]
	\centering
	\begin{minipage}[htbp]{7.5cm}
		\centering
		\caption{R类信号源识别结果}
		\begin{tabular}{ccc}
			\toprule
			信号源批次	&	信号源名称	&	得分			\\
			\midrule
			1011			& 	DRZ-60  		&	0.984037276 	\\
			1012		&	DRQ-7646	&	0.987897002	\\
			1013		&	DRL-75		&	0.967058036	\\
			1041		&	DRZ-59		&	0.978429748	\\
			1042		&	DRJ-72		&	0.982316416	\\
			1043		&	DRQ-7646	&	0.98102878	\\
			1071		&	DRU-7862	&	0.993292874	\\
			1072		&	DRB-7A		&	0.968448844	\\
			1073		&	DRL-49		&	0.960267684	\\
			1101			&	DRZ-60		&	0.971984401	\\
			1102 		&	DRK-89		&	0.98550102	\\
			1103 		&	DRN-8799I7	&	0.977801401	\\
			1131 		&	DRN-8799I7	&	0.976011509	\\
			\bottomrule
		\end{tabular}
	\end{minipage}
	\begin{minipage}[htbp]{7.5cm}
		\centering
		\begin{minipage}[htbp]{7.5cm}
			\centering
			\caption{L1类信号源识别结果}
			\begin{tabular}{ccc}
				\toprule
				信号源批次	&	信号源名称	&	得分			\\
				\midrule
				1014		&	DL9-Q9		&	0.997071899	\\
				1044		&	DL9-Q3		&	0.997489923	\\
				1074		&	DL9-Q8		&	0.992892631	\\
				1104			&	DL9-P3		&	0.993948117	\\
				\bottomrule
			\end{tabular}
		\end{minipage}
		\begin{minipage}[htbp]{7.5cm}
			\centering
			\caption{L2类信号源识别结果}
			\begin{tabular}{ccc}
				\toprule
				信号源批次	&	信号源名称	&	得分			\\
				\midrule
				1015		&	DL5-Q5		&	0.993011041	\\
				\bottomrule
			\end{tabular}
		\end{minipage}
		\begin{minipage}[htbp]{7.5cm}
			\centering
			\caption{A类信号源识别结果}
			\begin{tabular}{ccc}
				\toprule
				信号源批次	&	信号源名称	&	得分		\\
				\midrule
				1106			&	DA-P5A		&	1		\\
				\bottomrule
			\end{tabular}
		\end{minipage}
	\end{minipage}
\end{table}
\newpage
\subsection{模型的检验}
模型求解结果中每一船舶最优解的得分虽然接近1,但如果所有解都普遍接近1,则识别结果的可信度会令人怀疑。
\\\indent 为此,计算每一船舶所有解得分的平均值,可见求得的最优解得分显著超过平均值。
\begin{table}[htbp]
	\centering
	\caption{得分比较}
	\begin{tabular}{ccc}
		\toprule
		目标编号	&	最优得分	&	平均得分		\\
		\midrule
		101	&	0.930799861	&	4.12273E-05	\\
		104	&	0.940527104	&	1.46971E-05	\\
		107	&	0.917167393	&	9.04712E-05	\\
		110	&	0.930959406	&	8.85545E-05	\\
		113	&	0.976011509	&	0.06231429	\\
		\bottomrule
	\end{tabular}
\end{table}
