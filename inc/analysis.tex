\clearpage
\section{问题分析}
\subsection{假设}
\begin{enumerate}
	\item 船舶运动路径远远小于地球平均半径,可将地球球面视为平面;
	\item 为确保距离的准确性,假设传感器是固定的;
	\item 由于电磁信号传播速度接近光速,且传感器监测范围有限,因此可以忽略时间信号与船体运动之间的关系。同时假设接收器数据预处理时间足够的快,可以认为同一传感器的时间测量是准确的;
	\item 由于传感器原理结构的未知性,系统误差无从分析,本文忽略不计。
	\item 假设同一传感器对于同一信号源的同一种数据信息服从相同的正态分布。
\end{enumerate}
\subsection{标记}
\begin{table}[htbp]
	\centering
	\caption{变量的定义}
	\begin{tabular}{cc}
		\toprule
		变量				&	定义												\\
		\midrule
		\(\bm{x}_0\)		&	坐标测量值										\\
		\(\bm{x}\)			&	表征通过测量确定坐标真实值概率分布的随机变量		\\
		\(\bm{e}\)			&	误差												\\
		\(\Delta t\)		&	与上次测量间隔的时间								\\
		\(\bm{x}\bm{'}\)	&	表征通过推测确定坐标真实值概率分布的随机变量		\\
		\(\bm{v}\)			&	表征通过测量确定速度真实值概率分布的随机变量		\\
		\(\bm{x}\bm{'}_0\)	&	上次表征通过综合确定坐标真实值概率分布的随机变量	\\
		\(K\)				&	Kalman增益										\\
		\(d\)				&	最小包容圆柱直径									\\
		\(p\)				&	指标得分函数										\\
		\(r\)				&	未知信号源										\\
		\(R\)				&	数据库中的已知信号源								\\
		\(s\)				&	未知船舶											\\
		\(S\)				&	数据库中的已知船舶								\\
		\bottomrule
	\end{tabular}
\end{table}
\subsection{分析}
\subsubsection{题一}
考虑到具有相同“信号源批号”和“传感器类型”的数据,可视为来自同一信号源。附件一中传感器类型相同,因此按照信号源批号将数据分为四组。
\\\indent 误差分为3种:粗大误差、系统误差、随机误差。
\begin{description}
	\item[粗大误差]粗大误差可以完全消除。主要来源于测量人员。对粗大误差的鉴别通常有PauTa\index{PauTa}准则、Grubbs\index{Grubbs}准则、Chauvenet\index{Chauvenet}准则和Dixon\index{Dixon}准则。根据参考文献0,在采样数据较少的情况下适用PuaTa准则。本模型选用PauTa准则消除粗大误差。
	\item[系统误差]系统误差可以完全消除。主要来源于测量工具、测量方法和测量环境。题1未指出传感器的具体名称,其工作原理、工作环境可能导致的系统误差也无从分析。故忽略不计。
	\item[随机误差]随机误差只能减小不能完全消除。主要来源于测量环境。对随机误差的消除有PID调节、Kalman滤波等。由于船舶航行中受到各种不确定因素的影响如船舶结构、海洋环境等,这种误差是大小和正负都不确定的因此假设随机误差产生的影响服从正态分布。针对正态分布的随机误差,选用Kalman\index{Kalman}滤波对随机误差进行减小\cite{Kalman} 。
\end{description}
\par 数据信息分为信号特征信息和时空位置信息。
\begin{description}
	\item[信号特征信息]通过观察附件中已知数据可知信号特征信息变化很小且数值分布集中,因此假设认为信号特征信息随机误差很小,只需剔除粗大误差即可。
	\item[时空位置信息]考虑到时间信息与船体运动与环境无关,因此假设时间信息是准确的,只需对经纬度信息进行处理。先利用PauTa准则剔除位置信息粗大误差,再依据Kalman滤波综合测量与预测结果获得更为精确的位置信息。
\end{description}
\subsubsection{题二}
每一个信号源都是多个数据点构成的点集,每一只船都是多个信号源构成的集合。
\begin{description}
	\item[降维]因为传统的聚类只能对点集聚类而无法实现对多个点集构成的集合聚类,所以我们必须要提取点集具有特征性的标志数据,将对多个点集构成的集合聚类转化为对提取到的新数据点构成的集合聚类。即实现数据的降维。
	\item[聚类]本文使用K均值聚类算法\cite{clustering} 。
\end{description}
\subsubsection{题三}
我们将识别分为2个阶段:信号源识别和船舶识别。
\begin{description}
	\item[信号源识别]考虑到信号数据有两种,一种是需要与数据库中完全符合的,第二种是允许与数据库信息存在误差的。我们分别称为匹配指标和识别指标。若信号源匹配指标数据与数据库中完全符合则得分为1,否则为0;识别指标得分定义为信号源、数据库两者中较小的数据除以较大的数据以此表征相似程度,得分越趋近于1,两者数据越接近。我们建立信号源识别得分矩阵,矩阵中任一值为传感器所有指标得分的乘积;
	\item[船舶识别]根据信号源识别得分矩阵我们建立船舶识别得分矩阵,矩阵中任一值为船舶所有指标得分的乘积,表征船舶与数据库识别成功的概率,值越接近于1,概率越高。
\end{description}
\subsubsection{题四}
综合利用上述三个模型对于仿真环境中的数据进行目标识别,判断出目标数量和目标名称。
\begin{enumerate}
	\item 首先是利用模型一对仿真环境中的数据进行误差处理;
	\item 然后再利用模型二通过信号源判断出目标数量;
	\item 最后利用模型三进行目标识别。
\end{enumerate}
