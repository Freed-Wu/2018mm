\clearpage
\setcounter{section}{-1}
\section{引言}
\subsection{问题背景}
中国拥有299.7万平方公里的海洋面积,约为陆地面积的三分之一。然而我国是传统的大陆国家,长期以来对于海洋权益并不重视,对于领海并没有有效的行使主权,同时为了维护现阶段和平的国际环境,我国对于邻海争端只能选择暂时“搁置争议”,产生了当前面对岛屿被侵占时缺乏有力回击的被动局面。党的十六大上就提出“实施海洋开发”,要努力增强海洋意识,弘扬海洋文化,发展海洋经济。实施海洋开发战略,首要是维护国家海洋权益,对海域实施有效管理。然而海洋不同于陆地,传统的监控方式往往难以实现全面覆盖。近年来随着科技水平的提高,通过多种传感器对海上目标进行侦察监控、身份识别逐渐成为海域管理的重要技术手段。
\\\indent 各类船舶是海上目标的主体,船舶在航行过程中,基于航行、定位、通信等需求,在不同的时间点会发射或反射多种类型的电磁信号,这些被传感器所接收,成为识别目标身份的主要依据。
\subsection{问题信息}
\begin{enumerate}
	\item 电磁信号的信号源可分为4类,简记为R类、L1类、L2类、A类,相应的也有4类接收信号的传感器;
	\item 信号数据一般包含两个部分,一是时空位置信息(“时间”、“经度”、“纬度”),二是信号特征信息;
	\item 具有相同“信号源批号”和“传感器类型”的数据,可视为来自同一信号源;
	\item 由于对信号数据的不同处理记录方式,信号数据分为两大类。一是识别过程中必须完全与信号源数据库中记录相符合的数据,包括R类信号数据中“RA\_1\_1”、“RA\_2\_1”、“RA\_3\_1”、\\“RA\_4\_1”和A类信号数据中“MMSI编号”;二是识别过程中允许与数据库存在偏差的数据,包括R类信号中“RA\_1\_2”、“RA\_2\_2”、“RA\_3\_2”、“RA\_4\_2”,L1类信号中“L1A\_1”、“L1A\_2”、“L1A\_3”,L2类信号中“L2A\_1”、“L2A\_2”、“L2A\_3”和A类信号中“AA\_1”、“AA\_2”;
	\item 一般而言,识别过程中偏差越小,信号数据越有可能识别为数据库中对应的信号源。
\end{enumerate}
\subsection{问题重述}
\begin{enumerate}
	\item 由于噪声、干扰、传感器校准等因素的影响,传感器在获取信号和数据预处理的过程中不可避免的会出现误差,请针对时空位置信息和信号特征信息分别建立减小误差的数学模型,利用附件1的数据检验你们的模型是否有效;
	\item 同一信号源的时空位置信息可构成一条运动轨迹。由于船舶发出的多种类型的信号可能被不同的传感器监测到,因此在识别过程中,同一目标可能具有多条运动轨迹。对船舶类目标,R类信号源至多有3个,L1类、L2类、A类信号源至多有1个。目标识别过程中需要判断哪些运动轨迹属于同一个目标,请建立数学模型解决这一问题,针对附件2的数据给出判断结果,分析结果的可信程度;
	\item 经过数据处理之后,对目标的识别一般分为两个步骤,一是对信号源的识别,二是对目标的识别。附件3和附件4分别给出部分已知船舶和信号源数据库,附件5给出一批人工处理后的信号特征信息,其中具有同一“目标编号”的信号源可视为来自同一船舶目标。请建立根据信号特征信息识别目标船舶的数学模型,利用该模型识别附件5中的信号来自附件3数据库中的哪些目标船舶,并分析识别结果的可信程度;
	\item 综合应用上述问题中建立的模型,以及附件3、附件4的数据库,对附件6的信号数据进行自动识别,判断共有多少个目标,各目标的识别结果是什么,并分析结果的可信程度。
\end{enumerate}
