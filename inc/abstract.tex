\setcounter{page}{1}
\maketitle
\thispagestyle{empty}
\begin{abstract}
	我国近年来逐渐重视开始海洋,为了维护海洋权益,针对海上复杂情况行之有效的目标识别与监控系统的建立显得尤为重要。本文针对海上目标识别问题,将其分为信号误差处理、目标信号源轨迹分析并确认其归属、信号源识别和船舶识别三个步骤,以此最终确定目标数量和身份。
	\\\indent 针对问题一,建立误差处理模型减小误差。误差分为粗大误差、系统误差、随机误差。由于系统误差往往与传感器原理、结构有关而传感器是未知的,因此假设系统误差为一微小常量忽略不计。针对粗大误差我们利用PauTa准则剔除粗大误差的数据;针对随机误差,利用Kalman滤波模型综合测量与推测的结果通过不断迭代减小随机误差。针对传感器信号特征信息,由于信号特征信息数值相差很小且分布集中,因此可认为信号特征信息随机误差很小,一般为定值,故仅需利用PauTa准则剔除粗大误差;针对时空位置信息,考虑到数据时间间隔和船舶实际加速度的有限性,首先利用PauTa准则去除加速度过大的粗大误差数据,然后再通过Kalman滤波算法减小随机误差。误差处理模型是之后聚类和识别模型的重要基础。
	\\\indent 针对问题二,进行信号源轨迹分析并确认其归属。考虑到同一船舶多信号源的可能性,我们选用最小包容圆柱直径表征两信号源轨迹的平行度和间距,利用最小包容区域直径衡量两信号源属于同一船舶的概率,直径越小,概率越大。在此基础上进行聚类。由于实际船舶航行时需要保持安全距离,因此我们对聚类结果加上安全距离的约束限制,经多次匹配得到最优聚类。
	\\\indent 针对问题三,进行目标识别。针对信号源识别,因为数据的指标分为需要与数据库完全相同和允许有误差的两类,所以对匹配指标和识别指标的评分分类讨论。匹配指标得分只有0和1两种取值,当且仅当完全相同,得分才为1;识别指标得分取值范围为[0,1],误差越小指标越接近1;最后将所有指标相乘获得两信号源识别得分,表征识别正确概率,越接近于1概率越大;针对船舶识别,我们定义船舶识别得分为船舶的信号源数据与数据库数据识别的得分之积,取值范围也是[0,1]。识别得分最高的结果就是最可能的结果。
	\\\indent 针对问题四,我们综合利用上述三个模型完成最终识别。由于仿真环境中信号源极多导致情况复杂,首先我们进行误差处理,防止误差过大影响最终识别的结果;其次我们利用模型二完成信号源轨迹分析与归属确定,找到属于同一船舶的信号源记为一组数据,求出目标船舶的数量,并减小最终识别的计算量;最后利用模型三完成信号源的识别与船舶识别,求出识别得分函数,判断出目标船舶的身份。
	\\\textbf{关键词:}PauTa准则;Kalman滤波;最小包容原则;多维标度法;K均值聚类;海上目标识别。
\end{abstract}
\clearpage
%\begin{center}
%	\normalsize\textbf{Summary}
%\end{center}
%\par\small
%\begin{spacing}{0.5}
%	In recent years, China has attached great importance to the beginning of the ocean. In order to safeguard the rights and interests of the ocean, it is very important to establish an effective target recognition and monitoring system for the complex sea situation. Aiming at the problem of target recognition at sea, this paper divides it into three steps: signal error processing, target signal source ascription unified and trajectory analysis, signal source recognition and ship recognition, in order to determine the number, identity and trajectory of the target.
%	\\\indent set up an error handling model to reduce errors. The error is divided into gross error, systematic error and random error. Because the system error is often related to the sensor's principle and structure, the sensor is unknown, so the system error is assumed to be negligible. For the rough error, we use the PuaTa criterion to eliminate the rough error data. For the random error, the Calman filter model is used to measure and speculate, and the random error is reduced by continuous iteration. In view of the characteristic information of sensor signal, the random error of signal feature information is very small because of the small and distributed concentration of the signal feature information. It only needs to eliminate the rough error by using the PauTa criterion. In view of the spatiotemporal location information, the data time interval and the limited ship actual acceleration are considered. The Pauta criterion removes the gross error data with excessive acceleration, and then reduces the random error through Calman filtering algorithm. The error handling model is an important foundation for subsequent clustering and recognition models.
%	\\\indent aims at problem two. Considering the possibility of the multi signal source of the same ship, we select the minimum containment region diameter to characterize the relationship between the parallelism and the distance between the multi signal source trajectories, and use the minimum containment region diameter to indicate the probability of the multi signal source belonging to the unified ship, reducing the attribution probability to the one dimension variable function and reducing the operation amount; due to the actual ship navigation. We need to maintain a safe distance, so we restrict the clustering results to a safe distance and restrict the optimal clustering.
%	\\\indent for target three, target recognition and database matching. According to the signal source recognition, because the data is divided into two types of data which need to be complete with the database and allow the error of the error, the matching index and the identification index are separately discussed. The matching index is only 0,1 two kinds of values, if the index is 1, the range of recognition index is [0,1], the smaller the error is, the closer to 1. Finally, the multisignal source recognition matrix is multiplied by all the indexes. The product characterizing the recognition accuracy, the closer to the 1 probability, the better the identification of ship identification, we define ship recognition. The value of the matrix is the ratio of the ship's multi signal source data to the database data, and the range of the value is also [0,1]. The multisignal source product consists of the ship identification matrix. For the recognition matrix, we add a constraint condition that any row or column can only select a non zero value to reduce the probability and the number of comparisons, and get the most accurate identification results.
%	\\\indent for problem four, we use the above three models to complete the final identification. Because of the complex situation of multiple ship multi signal sources in the simulation environment, we first carry out the error processing to prevent the error too large to affect the final recognition result. Secondly, we use the model two to complete the signal source trajectory analysis and ascription, and find the signal source belonging to the same ship as a set of data and reduce the final recognition calculation. Finally, the identification and propagation recognition of the signal source are completed by model three, the identification matrix is established and the corresponding relationship is established with the database to determine the number, identity and movement state of the ship. 
%	\\\textbf{Keywords: }Sea target recognition; PauTa criterion; Kalman filtering; Minimum containment principle; Multidimensional scaling method; K-means clustering.
%\end{spacing}
%\normalsize
%\clearpage
