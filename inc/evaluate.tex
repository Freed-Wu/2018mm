\clearpage
\section{模型的评价与改进}
\subsection{优点}
\begin{enumerate}
	\item 针对不同类型的数据误差进行不同的处理,有效滤除了信号中粗大误差,减小了随机误差,为后面的模型的处理分析奠定了良好基础。适用于实际情况中多目标多信号源的复杂数据量;
	\item 聚类模型考虑全面且贴近实际情况,构造了最小包容区域直径有效地表征了信号源轨迹关系,辅之安全距离的判断,贴合实际情况。对传统聚类模型的改进有效控制了计算量复杂程度,且聚类结果准确;
	\item 识别模型中针对不同数据特点构建了不同的指标,且指标取值均在[0,1]之内,方便后续指标相乘进行指标综合统一建立识别矩阵,减小识别计算量,有效简化了识别过程,且十分贴合实际情况。
\end{enumerate}
\subsection{缺点}
\begin{enumerate}
	\item 由于缺乏传感器基本原理和构造的了解,没有对信号数据中的系统误差做出针对性的消除。且由于缺乏传感器正常工作信号范围,滤除粗大误差中可能存在错误情况;
	\item 实际情况中船只航行安全距离与船只尺寸参数和运动状态有关,由于缺乏实际船只尺寸,因此安全距离的判别中与实际情况是有出入的;
	\item 考虑到实际状况更多目标和传感器的复杂状况,单纯的一一比对数据库的算法计算量过大,并不适用实际情况。
\end{enumerate}
\subsection{改进}
\begin{enumerate}
	\item 实际情况中在已知传感器选型、原理和机构的情况下,可以进一步消除系统误差,进一步保证数据信号准确性。
	\item 考虑到船只实际尺寸参数与安全距离之间的联系,可以进一步针对实际船只情况建立安全距离标准,合理优化安全距离的判断条件,既简化了重复选择判断的计算量,也更符合实际安全保障。
	\item 识别中的错误率是难免的。针对民用和军事情况可以建立不同的错误率标准,针对不同情况,识别模型可以进一步优化,民用模型可以进一步简化模型,减少匹配指标数,仅考虑MMSI编号这类唯一性指标即可,可以大大简化计算量。
	\item 考虑到实际广阔海域多目标多传感器的复杂情况,可以优化比对数据库的算法,比如建立贝叶斯网络推理、证据理论或模糊判断等模型简化比对过程,控制识别成功率的同时大大减小计算量。
\end{enumerate}
\subsection{推广}
\begin{enumerate}
	\item 以中国目前面对东海钓鱼岛严峻态势为例,考虑建立海上目标自动识别的应用;
	\item 海上目标识别的重要意义在于对于广阔海域进行实时、半实时的监控。通过建立国内已知船舶编号数据库和信号源数据库,可以完成领海内海上目标识别,判断出友方和未知方。对于陌生方,可以通过不同的传感器判断出未知船只的运动状态(距离、速度、运动趋势)和船只特征(船只类型、电子装备情况),辅之以基于Bayes网络或者证据理论的相关知识可以完成对未知目标类型和威胁程度的判断,有效保障了领海海域内的安全\cite{Bayes} ;
	\item 海上目标识别与目标状态估计相结合完成了多信号的整合分析,构成了态势分析和威胁估计的基础,是安全分析与决策的情报基础。为了保护钓鱼岛周边中国渔民的正常经济活动和应对日本时常进行登岛挑衅活动,海上目标识别系统显得尤为重要。
\end{enumerate}
\subsection{展望}
\begin{enumerate}
	\item 信息处理与判断是战场态势分析的基础。现代战争是海陆空多维战争,因此模型还可以进一步推广。考虑到现代战场广阔且复杂的环境会导致获取的信息是多信号类型且不精确的,因此首先要对各种信号进行误差处理,其次是多信号整合识别与匹配作为态势判断和作战决策的基础。优化模型添加基于证据理论的模糊识别和基于Bayes网络推理等相关知识完善信号源识别与威胁模式判断,可以获取实时战场态势作为战术目标决策的基础。优化模型有效提高了目标综合识别的正确率和实时性,针对多信号源、信号误差等关键问题提出了合理的解决方案\cite{identify} 。
\end{enumerate}
